\documentclass[10,portrait]{article}
\usepackage{lmodern}
\usepackage{amssymb,amsmath}
\usepackage{ifxetex,ifluatex}
\usepackage{fixltx2e} % provides \textsubscript
\ifnum 0\ifxetex 1\fi\ifluatex 1\fi=0 % if pdftex
  \usepackage[T1]{fontenc}
  \usepackage[utf8]{inputenc}
\else % if luatex or xelatex
  \ifxetex
    \usepackage{mathspec}
  \else
    \usepackage{fontspec}
  \fi
  \defaultfontfeatures{Ligatures=TeX,Scale=MatchLowercase}
\fi
% use upquote if available, for straight quotes in verbatim environments
\IfFileExists{upquote.sty}{\usepackage{upquote}}{}
% use microtype if available
\IfFileExists{microtype.sty}{%
\usepackage[]{microtype}
\UseMicrotypeSet[protrusion]{basicmath} % disable protrusion for tt fonts
}{}
\PassOptionsToPackage{hyphens}{url} % url is loaded by hyperref
\usepackage[unicode=true]{hyperref}
\PassOptionsToPackage{usenames,dvipsnames}{color} % color is loaded by hyperref
\hypersetup{
            pdftitle={Instructions for runnng the individual-based Dynamic Energy Budget Schistosoma population model (SIDEB) on Mac OSX},
            colorlinks=true,
            linkcolor=blue,
            citecolor=red,
            urlcolor=blue,
            breaklinks=true}
\urlstyle{same}  % don't use monospace font for urls
\usepackage[margin=1in]{geometry}
\usepackage[]{biblatex}
\usepackage{color}
\usepackage{fancyvrb}
\newcommand{\VerbBar}{|}
\newcommand{\VERB}{\Verb[commandchars=\\\{\}]}
\DefineVerbatimEnvironment{Highlighting}{Verbatim}{commandchars=\\\{\}}
% Add ',fontsize=\small' for more characters per line
\usepackage{framed}
\definecolor{shadecolor}{RGB}{248,248,248}
\newenvironment{Shaded}{\begin{snugshade}}{\end{snugshade}}
\newcommand{\KeywordTok}[1]{\textcolor[rgb]{0.13,0.29,0.53}{\textbf{#1}}}
\newcommand{\DataTypeTok}[1]{\textcolor[rgb]{0.13,0.29,0.53}{#1}}
\newcommand{\DecValTok}[1]{\textcolor[rgb]{0.00,0.00,0.81}{#1}}
\newcommand{\BaseNTok}[1]{\textcolor[rgb]{0.00,0.00,0.81}{#1}}
\newcommand{\FloatTok}[1]{\textcolor[rgb]{0.00,0.00,0.81}{#1}}
\newcommand{\ConstantTok}[1]{\textcolor[rgb]{0.00,0.00,0.00}{#1}}
\newcommand{\CharTok}[1]{\textcolor[rgb]{0.31,0.60,0.02}{#1}}
\newcommand{\SpecialCharTok}[1]{\textcolor[rgb]{0.00,0.00,0.00}{#1}}
\newcommand{\StringTok}[1]{\textcolor[rgb]{0.31,0.60,0.02}{#1}}
\newcommand{\VerbatimStringTok}[1]{\textcolor[rgb]{0.31,0.60,0.02}{#1}}
\newcommand{\SpecialStringTok}[1]{\textcolor[rgb]{0.31,0.60,0.02}{#1}}
\newcommand{\ImportTok}[1]{#1}
\newcommand{\CommentTok}[1]{\textcolor[rgb]{0.56,0.35,0.01}{\textit{#1}}}
\newcommand{\DocumentationTok}[1]{\textcolor[rgb]{0.56,0.35,0.01}{\textbf{\textit{#1}}}}
\newcommand{\AnnotationTok}[1]{\textcolor[rgb]{0.56,0.35,0.01}{\textbf{\textit{#1}}}}
\newcommand{\CommentVarTok}[1]{\textcolor[rgb]{0.56,0.35,0.01}{\textbf{\textit{#1}}}}
\newcommand{\OtherTok}[1]{\textcolor[rgb]{0.56,0.35,0.01}{#1}}
\newcommand{\FunctionTok}[1]{\textcolor[rgb]{0.00,0.00,0.00}{#1}}
\newcommand{\VariableTok}[1]{\textcolor[rgb]{0.00,0.00,0.00}{#1}}
\newcommand{\ControlFlowTok}[1]{\textcolor[rgb]{0.13,0.29,0.53}{\textbf{#1}}}
\newcommand{\OperatorTok}[1]{\textcolor[rgb]{0.81,0.36,0.00}{\textbf{#1}}}
\newcommand{\BuiltInTok}[1]{#1}
\newcommand{\ExtensionTok}[1]{#1}
\newcommand{\PreprocessorTok}[1]{\textcolor[rgb]{0.56,0.35,0.01}{\textit{#1}}}
\newcommand{\AttributeTok}[1]{\textcolor[rgb]{0.77,0.63,0.00}{#1}}
\newcommand{\RegionMarkerTok}[1]{#1}
\newcommand{\InformationTok}[1]{\textcolor[rgb]{0.56,0.35,0.01}{\textbf{\textit{#1}}}}
\newcommand{\WarningTok}[1]{\textcolor[rgb]{0.56,0.35,0.01}{\textbf{\textit{#1}}}}
\newcommand{\AlertTok}[1]{\textcolor[rgb]{0.94,0.16,0.16}{#1}}
\newcommand{\ErrorTok}[1]{\textcolor[rgb]{0.64,0.00,0.00}{\textbf{#1}}}
\newcommand{\NormalTok}[1]{#1}
\IfFileExists{parskip.sty}{%
\usepackage{parskip}
}{% else
\setlength{\parindent}{0pt}
\setlength{\parskip}{6pt plus 2pt minus 1pt}
}
\setlength{\emergencystretch}{3em}  % prevent overfull lines
\providecommand{\tightlist}{%
  \setlength{\itemsep}{0pt}\setlength{\parskip}{0pt}}
\setcounter{secnumdepth}{0}
% Redefines (sub)paragraphs to behave more like sections
\ifx\paragraph\undefined\else
\let\oldparagraph\paragraph
\renewcommand{\paragraph}[1]{\oldparagraph{#1}\mbox{}}
\fi
\ifx\subparagraph\undefined\else
\let\oldsubparagraph\subparagraph
\renewcommand{\subparagraph}[1]{\oldsubparagraph{#1}\mbox{}}
\fi

% set default figure placement to htbp
\makeatletter
\def\fps@figure{htbp}
\makeatother


\title{Instructions for runnng the individual-based Dynamic Energy Budget
\emph{Schistosoma} population model (SIDEB) on Mac OSX}
\author{Matthew Malishev\textsuperscript{1}* \& David J.
Civitello\textsuperscript{1}\\[2\baselineskip]\emph{\textsuperscript{1}
Department of Biology, Emory University, 1510 Clifton Road NE, Atlanta,
GA, USA, 30322}}
\date{}

\begin{document}
\maketitle

{
\hypersetup{linkcolor=black}
\setcounter{tocdepth}{4}
\tableofcontents
}
\newpage   

Date: 2018-11-16\\
R version: 3.5.0\\
*Corresponding author:
\href{mailto:matthew.malishev@gmail.com}{\nolinkurl{matthew.malishev@gmail.com}}\\
This document can be found at
\url{https://github.com/darwinanddavis/SchistoIBM/tree/master/mac}

~

R session info

\begin{verbatim}
R version 3.5.0 (2018-04-23)
Platform: x86_64-apple-darwin15.6.0 (64-bit)
Running under: OS X El Capitan 10.11.6

Matrix products: default
BLAS: /Library/Frameworks/R.framework/Versions/3.5/Resources/lib/libRblas.0.dylib
LAPACK: /Library/Frameworks/R.framework/Versions/3.5/Resources/lib/libRlapack.dylib

locale:
[1] en_US.UTF-8/en_US.UTF-8/en_US.UTF-8/C/en_US.UTF-8/en_US.UTF-8

attached base packages:
[1] stats     graphics  grDevices utils     datasets  methods   base     

loaded via a namespace (and not attached):
 [1] compiler_3.5.0  backports_1.1.2 magrittr_1.5    rprojroot_1.3-2 tools_3.5.0     htmltools_0.3.6
 [7] pillar_1.2.3    tibble_1.4.2    yaml_2.2.0      Rcpp_0.12.19    stringi_1.2.3   rmarkdown_1.10 
[13] knitr_1.20      stringr_1.3.1   digest_0.6.15   rlang_0.3.0.1   evaluate_0.10.1
\end{verbatim}

\newpage  

\subsection{Overview}\label{overview}

Follow the instructions to run the simulation model from \texttt{R} or
\texttt{RStudio}. All reports and bugs should be addressed to
\href{mailto:matthew.malishev@gmail.com}{\nolinkurl{matthew.malishev@gmail.com}}.

\subsection{Required files}\label{required-files}

Files required for running the simulation are outlined below and will be
automatically loaded from the
\href{https://github.com/darwinanddavis/SchistoIBM}{\emph{Schistosoma}
IBM Github page} when running the model:

\begin{verbatim}
DEB_IBM.R
DEB_INF_GUTS_IBM.nlogo
FullStarve_shrink_production2.Rda
IndividualModel_IBM.c
IndividualModel_IBM.so
IndividualModel_IBM.o
\end{verbatim}

\subsection{Set system to run C
toolchain}\label{set-system-to-run-c-toolchain}

\begin{enumerate}
\def\labelenumi{\arabic{enumi}.}
\tightlist
\item
  Follow the steps for compiling a toolchain and running C code in
  \texttt{R} outlined at
  \href{https://thecoatlessprofessor.com/programming/r-compiler-tools-for-rcpp-on-macos/}{Installing
  compiler toolchain for Mac OSX}. See also
  \protect\hyperlink{references}{{[}4{]}} if the above steps don't work.
  Thanks to the \url{www.thecoatlessprofessor.com}.
\end{enumerate}

\subsection{Java tests}\label{java-tests}

Diagnostics for testing you have the correct version of Java on your
computer. Tests are sequential (from
\protect\hyperlink{references}{{[}5{]}}).

\begin{enumerate}
\def\labelenumi{\arabic{enumi}.}
\tightlist
\item
  Open \texttt{R} and run the following code.
\end{enumerate}

\begin{Shaded}
\begin{Highlighting}[]
\CommentTok{# test java is working}
\KeywordTok{require}\NormalTok{(RCurl)}
\NormalTok{script <-}\StringTok{ }\KeywordTok{getURL}\NormalTok{(}\StringTok{"https://raw.githubusercontent.com/darwinanddavis/SchistoIBM/master/mac/java_test.R"}\NormalTok{, }\DataTypeTok{ssl.verifypeer =} \OtherTok{FALSE}\NormalTok{)}
\KeywordTok{eval}\NormalTok{(}\KeywordTok{parse}\NormalTok{(}\DataTypeTok{text =}\NormalTok{ script))}
\KeywordTok{capture.output}\NormalTok{(}\KeywordTok{errorlist}\NormalTok{()) }\CommentTok{# read error message from java_test.R output}
\end{Highlighting}
\end{Shaded}

If you get an \texttt{R} error message e.g.
\texttt{Error\ in\ .jnew\ ...} or an error beginning with
\texttt{Failed\ Test\ ...}, move onto
\protect\hyperlink{install-jgr}{``Install JGR''}. Otherwise, continue to
\protect\hyperlink{run-rnetlogo}{Run RNetLogo}.

\hypertarget{run-rnetlogo}{\subsection{\texorpdfstring{Run
\texttt{RNetLogo}}{Run RNetLogo}}\label{run-rnetlogo}}

Load the `DEB\_IBM.R' file into your \texttt{R} session. Follow the
instructions to load the model and execute the simulation.

If you see the following \texttt{rJava} error in your \texttt{R}
session, proceed to step \protect\hyperlink{install-rjava}{Install
\texttt{rJava}}:\\
\hspace*{0.333em} ~

\begin{verbatim}
Error : .onLoad failed in loadNamespace() for 'rJava', details:
  call: dyn.load(file, DLLpath = DLLpath, ...)
  error: unable to load shared object ...
  .  
  .  
  .  
\end{verbatim}

~ ~

\hypertarget{install-rjava}{\subsection{\texorpdfstring{Install
\texttt{rJava}}{Install rJava}}\label{install-rjava}}

Open Terminal (\(Applications > Terminal\)) and run the following in
Terminal (sourced from \protect\hyperlink{references}{{[}1{]}} and
\protect\hyperlink{references}{{[}2{]}}) ~

\begin{Shaded}
\begin{Highlighting}[]
\FunctionTok{sudo}\NormalTok{ ln -s }\VariableTok{$(}\ExtensionTok{/usr/libexec/java_home}\VariableTok{)}\NormalTok{/jre/lib/server/libjvm.dylib /usr/local/lib}
\end{Highlighting}
\end{Shaded}

~ Now re-run the `DEB\_IBM.R' code in \texttt{R}.

The following error in \texttt{R} when executing \texttt{NLStart()}
means you have successfully navigated the \texttt{rJava} errors, but are
running into a Java issue that's preventing \texttt{R} from generating
the NetLogo GUI from \texttt{R}: ~ ~

\begin{verbatim}
java.awt.HeadlessException
    at java.awt.GraphicsEnvironment.checkHeadless(GraphicsEnvironment.java:204)
    at java.awt.Window.<init>(Window.java:536)
    at java.awt.Frame.<init>(Frame.java:420)
    at java.awt.Frame.<init>(Frame.java:385)
    at javax.swing.SwingUtilities$SharedOwnerFrame.<init>(SwingUtilities.java:1758)  
 .   
 .  
 .  
\end{verbatim}

~ For the above error, the \texttt{NLStart()} function should run
successfully with \texttt{gui=F}, which launches a headless GUI mode. ~

\begin{Shaded}
\begin{Highlighting}[]
\KeywordTok{NLStart}\NormalTok{(nl.path,}\DataTypeTok{gui=}\NormalTok{F,}\DataTypeTok{nl.jarname =} \KeywordTok{paste0}\NormalTok{(}\StringTok{"netlogo-"}\NormalTok{,ver_nl,}\StringTok{".jar"}\NormalTok{)) }\CommentTok{# open netlogo w/o gui}
\end{Highlighting}
\end{Shaded}

If the above steps don't work, move onto
\protect\hyperlink{install-jgr}{Install JGR}.

\hypertarget{install-jgr}{\subsection{Install JGR}\label{install-jgr}}

~ 1. Check you have the latest version of
\href{https://www.oracle.com/technetwork/java/javase/downloads/index-jsp-138363.html}{Java/Oracle}.

\begin{enumerate}
\def\labelenumi{\arabic{enumi}.}
\setcounter{enumi}{1}
\tightlist
\item
  Install JGR by running the following code in your \texttt{R} session
  (from \protect\hyperlink{references}{{[}3{]}}): ~
\end{enumerate}

\begin{Shaded}
\begin{Highlighting}[]
\KeywordTok{install.packages}\NormalTok{(}\StringTok{'JGR'}\NormalTok{,,}\StringTok{'http://www.rforge.net/'}\NormalTok{)}
\KeywordTok{library}\NormalTok{(JGR)}
\NormalTok{JGR}\OperatorTok{::}\KeywordTok{JGR}\NormalTok{()}
\end{Highlighting}
\end{Shaded}

Now load the `DEB\_IBM.R' file into your \texttt{JGR} session to run the
simulation model.

For up to date troubleshooting running \texttt{RNetLogo} on \texttt{R}
v. 6.0 or higher, see
\href{https://github.com/NetLogo/NetLogo/issues/1282}{Running Netlogo
6.0.+}.

\hypertarget{references}{\subsection{References}\label{references}}

\textsuperscript{1}
\href{https://stackoverflow.com/questions/30738974/rjava-load-error-in-rstudio-r-after-upgrading-to-osx-yosemite}{rJava
load error in RStudio/R after ``upgrading'' to OSX Yosemite}

\textsuperscript{2}
\href{http://paulklemm.com/blog/2015-02-20-run-rjava-with-rstudio-under-osx-10-dot-10/}{Run
rJava with RStudio using OSX 10.10}

\textsuperscript{3} \href{https://github.com/markush81/JGR}{Installing
JGR}

\textsuperscript{4}
\href{https://stackoverflow.com/questions/1616983/building-r-packages-using-alternate-gcc}{GCC
compiler in R}

\textsuperscript{5}
\href{https://stackoverflow.com/questions/14915898/rnetlogo-function-nlstart-fails-to-launch-gui}{Test
for working version of Java on your computer}

\printbibliography

\end{document}
