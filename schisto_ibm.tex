\documentclass[10,portrait]{article}
\usepackage{lmodern}
\usepackage{amssymb,amsmath}
\usepackage{ifxetex,ifluatex}
\usepackage{fixltx2e} % provides \textsubscript
\ifnum 0\ifxetex 1\fi\ifluatex 1\fi=0 % if pdftex
  \usepackage[T1]{fontenc}
  \usepackage[utf8]{inputenc}
\else % if luatex or xelatex
  \ifxetex
    \usepackage{mathspec}
  \else
    \usepackage{fontspec}
  \fi
  \defaultfontfeatures{Ligatures=TeX,Scale=MatchLowercase}
\fi
% use upquote if available, for straight quotes in verbatim environments
\IfFileExists{upquote.sty}{\usepackage{upquote}}{}
% use microtype if available
\IfFileExists{microtype.sty}{%
\usepackage[]{microtype}
\UseMicrotypeSet[protrusion]{basicmath} % disable protrusion for tt fonts
}{}
\PassOptionsToPackage{hyphens}{url} % url is loaded by hyperref
\usepackage[unicode=true]{hyperref}
\PassOptionsToPackage{usenames,dvipsnames}{color} % color is loaded by hyperref
\hypersetup{
            pdftitle={SchistoIBM},
            colorlinks=true,
            linkcolor=pink,
            citecolor=red,
            urlcolor=blue,
            breaklinks=true}
\urlstyle{same}  % don't use monospace font for urls
\usepackage[margin=1in]{geometry}
\usepackage[]{biblatex}
\IfFileExists{parskip.sty}{%
\usepackage{parskip}
}{% else
\setlength{\parindent}{0pt}
\setlength{\parskip}{6pt plus 2pt minus 1pt}
}
\setlength{\emergencystretch}{3em}  % prevent overfull lines
\providecommand{\tightlist}{%
  \setlength{\itemsep}{0pt}\setlength{\parskip}{0pt}}
\setcounter{secnumdepth}{0}
% Redefines (sub)paragraphs to behave more like sections
\ifx\paragraph\undefined\else
\let\oldparagraph\paragraph
\renewcommand{\paragraph}[1]{\oldparagraph{#1}\mbox{}}
\fi
\ifx\subparagraph\undefined\else
\let\oldsubparagraph\subparagraph
\renewcommand{\subparagraph}[1]{\oldsubparagraph{#1}\mbox{}}
\fi

% set default figure placement to htbp
\makeatletter
\def\fps@figure{htbp}
\makeatother


\title{SchistoIBM}
\author{Matthew Malishev\textsuperscript{1}* and David J
Civitello\textsuperscript{1}\\[2\baselineskip]\emph{\textsuperscript{1}
Department of Biology, Emory University, 1510 Clifton Road NE, Atlanta,
GA, USA, 30322}}
\date{}

\begin{document}
\maketitle

{
\hypersetup{linkcolor=black}
\setcounter{tocdepth}{4}
\tableofcontents
}
\subparagraph{}\label{section}

Date: 2018-09-24\\
R version: 3.5.0\\
*Corresponding author:
\href{mailto:matthew.malishev@gmail.com}{\nolinkurl{matthew.malishev@gmail.com}}\\
DOI: \url{https://github.com/darwinanddavis/SchistoIBM}

Model from
\href{https://onlinelibrary.wiley.com/doi/abs/10.1111/ele.12937}{Civitello
et al 2018 Bioenergetic theory predicts infection dynamics of human
schistosomes in intermediate host snails across ecological gradients}

\newpage  

\subsection{Overview}\label{overview}

Individual-based model of snail and schistosome population ecology
driven by energetic consequences of host and parasites in response to
resources and environmental temperature.

Population model of individual hosts (snails) and parasites
(schistosomes) in a lake system. Host and parasite population sizes and
production of different parasite life stages are driven by a population
model. Parasites are released into the habitat according to reproduction
rules of a full energy and mass budget model based on Dynamic Energy
Budget (DEB) theory. Hosts feed on resource patches in the habitat and
are exposed to schistosomes depending on the type of resource patch they
feed on.

The host population is size and age structured, where parasite loading
varies depending on the biomass of the host. Host biomass, including
energy reserves, maintenance, and reproductive potential, is modelling
through a DEB model.

The energetics model determines the intake and production of parasites.
The simulation runs over a year and emulates a lake system in Kenya
where schistosomes are found.

The energy reserves, somatic maintenance, and growth of hosts. The
reproductive capacity of hosts is the parasite biomass prodcuted.

\newpage  

\subsection{Environment}\label{environment}

\subsubsection{Resources}\label{resources}

\[
\frac
{dR}
{dt}
= r
  \left(
  1-\frac{R}{K}
  \right)
  R -
  \sum -i_{M}L^{2}f_{H}
\] ~ \(\sum -i_{M}L^{2}f_{H}\) = feeding by snails

\subsubsection{Miracidia}\label{miracidia}

\[
\frac
{dM}
{dt}
= M(t)
- \sum uptake 
- death
\] ~ \(M(t)\) = miracidia in environment\\
\(\sum uptake\) = infection of hosts by miracidia. \textbf{Not a DEB
parameter}

\subsubsection{Cercariae}\label{cercariae}

\[
\frac
{dC}
{dt}
= \sum \frac{dRp}{dt} 
- death
\] ~\\
\(\sum \frac{dRp}{dt}\) = cumulative release of cercariae (parasite
reproduction)

\newpage  

\subsection{Individual}\label{individual}

\subsubsection{Host}\label{host}

\paragraph{Host length}\label{host-length}

\[
  \frac 
    {d L}
    {d t}
    = \frac
      {gY_{VE}}
      {3\chi} 
        \cdot \left( \frac 
        {\kappa^{*} a_{M}e - (m_{V} + m_{R}E_{M}\delta)\chi L}
        {e + g}
        \right)
\]

\paragraph{Host reserve}\label{host-reserve}

\[
  \frac 
    {d e}
    {d t}
    = (f_{H} - e)
    \frac
      {a_{M}}
      {\chi E_{M}L} - p
        \left(
        \frac 
        {i_{PM}f_{P}}
        {E_{M}}
        \right)
\]

\paragraph{Host development (maturity)}\label{host-development-maturity}

\[
  \frac
  {d D}
  {d t}
  = \begin{cases}
      \text{if} \ D < D_{R}, & (1 - \kappa^{*})C - m_{D}D \\
      \text{if} \ D ≥ D_{R}, & 0
    \end{cases}
\]

\paragraph{Host reproduction}\label{host-reproduction}

\[
  \frac
  {d R_{H}}
  {d t}
  = \begin{cases}
      \text{if} \ D < D_{R}, & 0 \\
      \text{if} \ D ≥ D_{R}, & (1 - \kappa^{*})C - m_{D}D
    \end{cases}
\]

\subsubsection{Parasite}\label{parasite}

\paragraph{Parasite biomass (growth)}\label{parasite-biomass-growth}

\[
  \frac
  {d P}
  {d t}
  = P(Y_{PE}i_{PM}f_{P} (1-r_{P}) -m_{P})
\]

\paragraph{Parasite reproduction}\label{parasite-reproduction}

\[
  \frac
  {d R_{P}}
  {d t}
  = \gamma_{RP}Y_{PE}i_{PM}f_{P}r_{P}
\]

\paragraph{Damage density}\label{damage-density}

\[
  \frac 
    {d \delta}
    {d t}
    = \frac
      {\Theta} 
      {\chi L^{3}} 
      \cdot \frac
      {dR_{P}}
      {dt}
        + k_{R}(1-e)-k_{R}\delta-
        \frac
          {3\delta}
          {L}
          \cdot \frac 
            {dL}
            {dt}
\] ~ \(\frac {3\delta}{L}\cdot \frac {dL}{dt}\) = growth dilution.
Growth of host reduces damage density\\
\(\frac {\Theta} {\chi L^{3}}\) = damage intensity\\
\(k_{R}\) = damage repair rate (via reserve depletion)

\paragraph{Cumulative hazard for host}\label{cumulative-hazard-for-host}

\[
  \frac
  {d H}
  {d t}
  = h_{b}+h_{\delta} \ max(\delta - \delta_{0}, \ 0)
\] ~ \(\delta_{0}\) = linear damage density function\\
\(h_{b}\) = background hazard rate\\
\(h_{\delta}\) = hazard coefficient

\newpage  

\subsection{References}\label{references}

\href{https://onlinelibrary.wiley.com/doi/abs/10.1111/ele.12937}{Civitello
et al 2018 Bioenergetic theory predicts infection dynamics of human
schistosomes in intermediate host snails across ecological gradients}

\printbibliography

\end{document}
